% Options for packages loaded elsewhere
\PassOptionsToPackage{unicode}{hyperref}
\PassOptionsToPackage{hyphens}{url}
\PassOptionsToPackage{dvipsnames,svgnames,x11names}{xcolor}
%
\documentclass[
  12pt,
  letterpaper,
]{article}

\usepackage{amsmath,amssymb}
\usepackage{iftex}
\ifPDFTeX
  \usepackage[T1]{fontenc}
  \usepackage[utf8]{inputenc}
  \usepackage{textcomp} % provide euro and other symbols
\else % if luatex or xetex
  \usepackage{unicode-math}
  \defaultfontfeatures{Scale=MatchLowercase}
  \defaultfontfeatures[\rmfamily]{Ligatures=TeX,Scale=1}
\fi
\usepackage{lmodern}
\ifPDFTeX\else  
    % xetex/luatex font selection
  \setmainfont[]{Baskerville}
  \setsansfont[]{Futura}
\fi
% Use upquote if available, for straight quotes in verbatim environments
\IfFileExists{upquote.sty}{\usepackage{upquote}}{}
\IfFileExists{microtype.sty}{% use microtype if available
  \usepackage[]{microtype}
  \UseMicrotypeSet[protrusion]{basicmath} % disable protrusion for tt fonts
}{}
\makeatletter
\@ifundefined{KOMAClassName}{% if non-KOMA class
  \IfFileExists{parskip.sty}{%
    \usepackage{parskip}
  }{% else
    \setlength{\parindent}{0pt}
    \setlength{\parskip}{6pt plus 2pt minus 1pt}}
}{% if KOMA class
  \KOMAoptions{parskip=half}}
\makeatother
\usepackage{xcolor}
\usepackage[margin=1in]{geometry}
\setlength{\emergencystretch}{3em} % prevent overfull lines
\setcounter{secnumdepth}{-\maxdimen} % remove section numbering


\providecommand{\tightlist}{%
  \setlength{\itemsep}{0pt}\setlength{\parskip}{0pt}}\usepackage{longtable,booktabs,array}
\usepackage{calc} % for calculating minipage widths
% Correct order of tables after \paragraph or \subparagraph
\usepackage{etoolbox}
\makeatletter
\patchcmd\longtable{\par}{\if@noskipsec\mbox{}\fi\par}{}{}
\makeatother
% Allow footnotes in longtable head/foot
\IfFileExists{footnotehyper.sty}{\usepackage{footnotehyper}}{\usepackage{footnote}}
\makesavenoteenv{longtable}
\usepackage{graphicx}
\makeatletter
\def\maxwidth{\ifdim\Gin@nat@width>\linewidth\linewidth\else\Gin@nat@width\fi}
\def\maxheight{\ifdim\Gin@nat@height>\textheight\textheight\else\Gin@nat@height\fi}
\makeatother
% Scale images if necessary, so that they will not overflow the page
% margins by default, and it is still possible to overwrite the defaults
% using explicit options in \includegraphics[width, height, ...]{}
\setkeys{Gin}{width=\maxwidth,height=\maxheight,keepaspectratio}
% Set default figure placement to htbp
\makeatletter
\def\fps@figure{htbp}
\makeatother
\newlength{\cslhangindent}
\setlength{\cslhangindent}{1.5em}
\newlength{\csllabelwidth}
\setlength{\csllabelwidth}{3em}
\newlength{\cslentryspacingunit} % times entry-spacing
\setlength{\cslentryspacingunit}{\parskip}
\newenvironment{CSLReferences}[2] % #1 hanging-ident, #2 entry spacing
 {% don't indent paragraphs
  \setlength{\parindent}{0pt}
  % turn on hanging indent if param 1 is 1
  \ifodd #1
  \let\oldpar\par
  \def\par{\hangindent=\cslhangindent\oldpar}
  \fi
  % set entry spacing
  \setlength{\parskip}{#2\cslentryspacingunit}
 }%
 {}
\usepackage{calc}
\newcommand{\CSLBlock}[1]{#1\hfill\break}
\newcommand{\CSLLeftMargin}[1]{\parbox[t]{\csllabelwidth}{#1}}
\newcommand{\CSLRightInline}[1]{\parbox[t]{\linewidth - \csllabelwidth}{#1}\break}
\newcommand{\CSLIndent}[1]{\hspace{\cslhangindent}#1}

% -----------------------
% CUSTOM PREAMBLE STUFF
% -----------------------

% -----------------
% Title block stuff
% -----------------

% Abstract
\usepackage[runin]{abstract}
\renewcommand{\abstractnamefont}{\sffamily\small\bfseries}
\renewcommand{\abstracttextfont}{\sffamily\small}
\setlength{\absleftindent}{5pt}
\setlength{\absrightindent}{\absleftindent}

% Title
\usepackage{titling}
\pretitle{\par\begin{flushleft}\LARGE\sffamily\bfseries}
\posttitle{\par\end{flushleft}\vskip 10pt}

% Keywords
\newenvironment{keywords}
{\small\sffamily{\sffamily\small\bfseries{Keywords.}}}

% Authors
\usepackage{orcidlink}  % Create automatic ORCID icons/links
%\renewcommand{\and}{\end{tabular} \hskip 3em \begin{tabular}[t]{@{\hspace{0em}}l@{}}}
\preauthor{\begin{flushleft}
           \lineskip 1.5em}
\postauthor{\end{flushleft}}

% ------------------
% Section headings
% ------------------
\usepackage{titlesec}
\titleformat*{\section}{\Large\sffamily\bfseries\raggedright}
\titleformat*{\subsection}{\large\sffamily\bfseries\raggedright}
\titleformat*{\subsubsection}{\normalsize\sffamily\bfseries\raggedright}
\titleformat*{\paragraph}{\small\sffamily\bfseries\raggedright}

%\titlespacing{<command>}{<left>}{<before-sep>}{<after-sep>}
% Starred version removes indentation in following paragraph
\titlespacing*{\section}{0em}{2em}{0.1em}
\titlespacing*{\subsection}{0em}{1.25em}{0.1em}
\titlespacing*{\subsubsection}{0em}{0.75em}{0em}

% ------------------
% Headers/Footers
% ------------------
\usepackage{fancyhdr}
\pagestyle{fancy}
\fancyhf{}
\fancyhead[L,C]{}
\fancyhead[R]{\leftmark}
\fancyfoot[L,C]{}
\fancyfoot[R]{\thepage}
\renewcommand{\headrulewidth}{1pt}
\fancypagestyle{plain}{%
    \renewcommand{\headrulewidth}{0pt}%
    \fancyhf{}%
    \fancyfoot[R]{\thepage}%
}
\renewcommand\footnoterule{\rule{\linewidth}{0.1pt}\vspace{5pt}}

% ------------------
% Captions
% ------------------
\usepackage[labelfont=bf,labelsep=period]{caption}
\captionsetup[figure]{font=footnotesize,justification=raggedright,singlelinecheck=false,format=hang}


% ---------------------------
% END CUSTOM PREAMBLE STUFF
% ---------------------------
\usepackage{booktabs}
\usepackage{longtable}
\usepackage{array}
\usepackage{multirow}
\usepackage{wrapfig}
\usepackage{float}
\usepackage{colortbl}
\usepackage{pdflscape}
\usepackage{tabu}
\usepackage{threeparttable}
\usepackage{threeparttablex}
\usepackage[normalem]{ulem}
\usepackage{makecell}
\usepackage{xcolor}
\usepackage{dcolumn}
\makeatletter
\makeatother
\makeatletter
\makeatother
\makeatletter
\@ifpackageloaded{caption}{}{\usepackage{caption}}
\AtBeginDocument{%
\ifdefined\contentsname
  \renewcommand*\contentsname{Table of contents}
\else
  \newcommand\contentsname{Table of contents}
\fi
\ifdefined\listfigurename
  \renewcommand*\listfigurename{List of Figures}
\else
  \newcommand\listfigurename{List of Figures}
\fi
\ifdefined\listtablename
  \renewcommand*\listtablename{List of Tables}
\else
  \newcommand\listtablename{List of Tables}
\fi
\ifdefined\figurename
  \renewcommand*\figurename{Figure}
\else
  \newcommand\figurename{Figure}
\fi
\ifdefined\tablename
  \renewcommand*\tablename{Table}
\else
  \newcommand\tablename{Table}
\fi
}
\@ifpackageloaded{float}{}{\usepackage{float}}
\floatstyle{ruled}
\@ifundefined{c@chapter}{\newfloat{codelisting}{h}{lop}}{\newfloat{codelisting}{h}{lop}[chapter]}
\floatname{codelisting}{Listing}
\newcommand*\listoflistings{\listof{codelisting}{List of Listings}}
\makeatother
\makeatletter
\@ifpackageloaded{caption}{}{\usepackage{caption}}
\@ifpackageloaded{subcaption}{}{\usepackage{subcaption}}
\makeatother
\makeatletter
\@ifpackageloaded{tcolorbox}{}{\usepackage[skins,breakable]{tcolorbox}}
\makeatother
\makeatletter
\@ifundefined{shadecolor}{\definecolor{shadecolor}{rgb}{.97, .97, .97}}
\makeatother
\makeatletter
\makeatother
\makeatletter
\makeatother
\ifLuaTeX
  \usepackage{selnolig}  % disable illegal ligatures
\fi
\IfFileExists{bookmark.sty}{\usepackage{bookmark}}{\usepackage{hyperref}}
\IfFileExists{xurl.sty}{\usepackage{xurl}}{} % add URL line breaks if available
\urlstyle{same} % disable monospaced font for URLs
\hypersetup{
  pdftitle={Using Bio Signals to Predict Smoker Status},
  pdfauthor={Uri Guerra},
  pdfkeywords={health, data science, public health, tobbacco use},
  colorlinks=true,
  linkcolor={blue},
  filecolor={Maroon},
  citecolor={Blue},
  urlcolor={red},
  pdfcreator={LaTeX via pandoc}}


\title{Using Bio Signals to Predict Smoker Status\thanks{Thanks to
everyone for checking this out.}}
% subtitles do not seem to work with article class?
%%\subtitle{}

\author{
{\bfseries \normalsize Uri Guerra~\orcidlink{0000-0001-7237-8131}}%
\thanks{Corresponding author.} \\%
 \small University of Oregon, Sociology \\%
{\footnotesize \url{mguerra2@uoregon.edu}} \\\vspace{10pt}
}

\predate{}
\postdate{}
\date{}
\begin{document}

% for some reason this does not work in header
\renewcommand{\abstractname}{Abstract.}

\fancyhead[L]{Predicting Smoker Status}

% need to redefine this environment to get enough spacing of 
% bibliography after title
\renewenvironment{CSLReferences}[2] % #1 hanging-ident, #2 entry spacing
 {% don't indent paragraphs
  \vspace{10pt}
  \setlength{\parindent}{0pt}
  % turn on hanging indent if param 1 is 1
  \ifodd #1
  \let\oldpar\par
  \def\par{\hangindent=\cslhangindent\oldpar}
  \fi
  % set entry spacing
  \setlength{\parskip}{#2\cslentryspacingunit}
 }%
 {}

\maketitle
%\noindent \rule{\linewidth}{.5pt}
\begin{abstract}
An Abstract
\end{abstract}
\begin{keywords}
\def\sep{;\ }
health\sep data science\sep public health\sep 
tobbacco use
\end{keywords}
%\noindent \rule{\linewidth}{.5pt}
\ifdefined\Shaded\renewenvironment{Shaded}{\begin{tcolorbox}[boxrule=0pt, frame hidden, breakable, interior hidden, borderline west={3pt}{0pt}{shadecolor}, sharp corners, enhanced]}{\end{tcolorbox}}\fi

\hypertarget{introduction}{%
\section{Introduction}\label{introduction}}

Using various health indicator information classification models are
utilized to predict smoker status. As it stands more than 16 million
american are currently living with a disease that can be linked to
smoking {``Health {Effects} of {Smoking} and {Tobacco} {Use}''} (2022).
The use of a model like this is useful in public health especially when
identifying links between certain bio signals and smoking status. It
could lead to a better understanding of individual effects smoking has
on the body due to identifying which health factors are most closely
associated with smoking status.

\hypertarget{description-of-data}{%
\section{Description of Data}\label{description-of-data}}

The data for this project was collected from an online competition
hosted on the Kaggle website called, ``Binary Prediction of Smoker
Status using Bio-Signals'' \emph{Binary {Prediction} of {Smoker}
{Status} Using {Bio}-{Signals}} (n.d.). The training data was generated
from a deep learning model that was generated from the ``Smoker Status
Prediction using Bio-Signals'' data set \emph{Smoker {Status}
{Prediction} Using {Bio}-{Signals}} (n.d.). The data set consist of 22
different predictors that range from age and height to health indicators
like cholesterol level, hemoglobin and systolic pressure.

\hypertarget{model-description-and-fit}{%
\section{Model Description and Fit}\label{model-description-and-fit}}

Using the training data set three different models were created which
include a generalized linear mode, classification model with ridge
penalty, and a classification model with lass penalty. Despite the use
of different models there did not appear to be siginifcant difference in
model performance. The way the models were evaluated was through the use
of the Log Loss and AROC to evaluate the models, with the results that
can be seen below.

\begin{longtable}[]{@{}lll@{}}
\caption{Model Performance Evaluation}\tabularnewline
\toprule\noalign{}
Model & Log Loss & AROC \\
\midrule\noalign{}
\endfirsthead
\toprule\noalign{}
Model & Log Loss & AROC \\
\midrule\noalign{}
\endhead
\bottomrule\noalign{}
\endlastfoot
GLM & 0.491946 & 0.8383444 \\
Classification Model with Ridge Penalty & 0.4951833 & 0.8350144 \\
Classification Model with Lasso Penalty & 0.4919702 & 0.8382552 \\
\end{longtable}

Log Loss was chosen since it highly penalizes incorrect predictions, as
the models were created to classify whether predictors indicated a
non-Smoker vs.~a Smoker. AROC however, was chosen because 1) the Kaggle
competition the data set came from utilized AROC to evaluate performance
2) in addition to its particular success at evaluation binary
classification models since it measues the probability distribution of
false and true positives.

\hypertarget{data-visualization}{%
\section{Data Visualization}\label{data-visualization}}

\begin{figure}[!t]

{\centering \includegraphics{main_manuscript_files/mediabag/000012.png-resize=16}

}

\end{figure}

\begin{figure}[!t]

{\centering \includegraphics{main_manuscript_files/mediabag/000014.png-resize=15}

}

\end{figure}

\hypertarget{conclusion}{%
\section{Conclusion}\label{conclusion}}

\textbf{?@fig-logistic-lasso}

\hypertarget{bibliography}{%
\section*{References}\label{bibliography}}
\addcontentsline{toc}{section}{References}

\hypertarget{refs}{}
\begin{CSLReferences}{1}{0}
\leavevmode\vadjust pre{\hypertarget{ref-noauthor_binary_nodate}{}}%
\emph{Binary {Prediction} of {Smoker} {Status} using {Bio}-{Signals}}.
(n.d.). Retrieved December 8, 2023, from
\url{https://kaggle.com/competitions/playground-series-s3e24}

\leavevmode\vadjust pre{\hypertarget{ref-noauthor_health_2022}{}}%
Health {Effects} of {Smoking} and {Tobacco} {Use}. (2022). In
\emph{Centers for Disease Control and Prevention}.
\url{https://www.cdc.gov/tobacco/basic_information/health_effects/index.htm}

\leavevmode\vadjust pre{\hypertarget{ref-noauthor_smoker_nodate}{}}%
\emph{Smoker {Status} {Prediction} using {Bio}-{Signals}}. (n.d.).
Retrieved December 8, 2023, from
\url{https://www.kaggle.com/datasets/gauravduttakiit/smoker-status-prediction-using-biosignals}

\end{CSLReferences}



\end{document}
